%\markright{\mititulo \hspace{5cm} \thepage}
\pagestyle{fancy}
\section*{Resumen}
\addcontentsline{toc}{section}{\protect\numberline{\thesection} RESUMEN}%

\noindent 
Las ruedas de reacción (Reaction Wheels, RW) son dispositivos ampliamente usados en los sistemas de control de actitud satelital debido a su precisión de orientación. Sin embargo, presentan un fenómeno de saturación debido a la acumulación de momento angular. Esta acumulación conlleva a que alcancen su límite de velocidad de rotación e impidan el intercambio de momento con el cuerpo del satélite para garantizar la estabilidad. Debido a esto, es de particular interés estudiar técnicas de desaturación de estos dispositivos empleando otros actuadores de control como los magnetorquers, los cuales, por medio de su interacción con campos magnéticos, generan un torque que contribuye a la desaceleración de las RW. Para ello, se propone un análisis computacional, que parta de un modelo dinámico basado en el CubeSat de entrenamiento EyasSat . Una vez obtenido dicho modelo, se propone realizar una comparación de diferentes controladores de actitud, con la capacidad de desaturar las RW, mediante índices de desempeño relacionados con el consumo energético, el tiempo de respuesta y el error en estado estable. A su vez se evaluará el rendimiento en diferentes escenarios al modificar parámetros orbitales e incluir fenómenos del medio ambiente espacial como la variación del campo magnético terrestre.

%El resumen permite identificar la esencia del escrito, mencionando brevemente el objetivo y la metodología, así como los resultados y las conclusiones (mínimo 150, máximo 250 palabras).


%Las palabras clave son los términos, materias y terminología que hacen posible describir y recuperar un documento en una disciplina específica. Pregúntese, por ejemplo: ¿con qué palabras puede un usuario de Internet recuperar mi documento? ¿Cuáles son los términos con los que mis colegas abordan esta temática? 3-7 palabras clave.
\vspace{1cm}\textbf{\textit{Palabras clave ---} Desaturación de ruedas de reacción, Sistema de determinación y control de actitud, Cubesats, EyasSat, Estrategias de Control}


%Miembros de la Facultad de Ciencias Económicas incluyen Clasificación JEL de la American Economic Association, disponible en https://bit.ly/3bdELGL
%Ejemplo de formato después de palabras clave:
%Clasificación JEL: Q01, Q13, Q16

\newpage
%----------------------------------

\section*{Abstract}
\addcontentsline{toc}{section}{\protect\numberline{\thesection} ABSTRACT}%

%No utilices traductores automáticos en línea, pues no tienen la capacidad de interpretar términos académicos y científicos. Es buena idea asesorarse de un traductor profesional. 

%El abstract es el mismo resumen pero en idioma inglés. Conserva la misma extensión o aproximada, es decir, mínimo 150 y máximo 250 palabras.

\noindent Reaction Wheels (RW) are widely used devices in satellite attitude control systems due to their orientation accuracy. However, they present a saturation phenomenon due to the accumulation of angular momentum. This accumulation leads them to reach their rotational speed limit and prevent the exchange of momentum with the satellite body to ensure stability. As a result, it is of particular interest to study unloading techniques using auxiliar control actuators such as magnetorquers, which, through their interaction with magnetic fields, generate a torque that contributes to the deceleration of the RW. For this purpose, a computational analysis is proposed, starting from a dynamic model based on the EyasSat training CubeSat. Once this model is obtained, a comparison of different attitude controllers with the ability to desaturate the RW is proposed. Performance indexes related to energy consumption, response time and steady state error are taken into account. At the same time, the performance will be evaluated in different scenarios by modifying orbital parameters and including space environment phenomena such as the variation of the Earth's magnetic field.


%Las keywords son las mismas palabras clave pero en inglés. Dependiendo de la disciplina, indague en bases de datos y plataformas cuáles son las traducciones al inglés de sus palabras clave, tales como IEEE Taxonomy (ingenierías IEEE http://goo.gl/nf29xl) o en diccionarios y tesauros especializados.  
\vspace{1cm}\textbf{\textit{Keywords ---}  Reaction Wheel Unloading, Attitude Determination and Control System, Cubesats, EyasSat, Control Strategies}

%Miembros de la Facultad de Ciencias Económicas incluyen Clasificación JEL de la American Economic Association, disponible en https://bit.ly/3bdELGL
%Ejemplo de formato después de keywords:
%JEL Classification: Q01, Q13, Q16

\newpage
%----------------------------------

